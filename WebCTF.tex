\documentclass{article}
\usepackage{xeCJK} % 使用 xeCJK 宏包来支持中文
\setCJKmainfont{PingFang SC}
\usepackage{listings}
\usepackage{xcolor}
\usepackage{amsmath}

\lstset{
    basicstyle=\ttfamily,
    keywordstyle=\color{blue},
    commentstyle=\color{green},
    stringstyle=\color{red},
    showstringspaces=false,
    breaklines=true,
    frame=lines,
    backgroundcolor=\color{lightgray},
    numbers=left,
    numberstyle=\tiny\color{gray},
    stepnumber=1,
    numbersep=5pt,
    tabsize=4,
    captionpos=b,
}

\title{常见Web安全漏洞及其修复}
\author{GeoffreyWang}
\date{\today}

\begin{document}

\maketitle

\section{SQL注入 (SQL Injection)}

\subsection{漏洞描述}
SQL注入是一种通过将恶意SQL代码注入到查询语句中,来操控数据库的攻击方式。这种攻击通常利用了应用程序直接将用户输入嵌入到SQL查询中的漏洞。

\subsection{攻击方式}
攻击者可能通过在输入字段中插入恶意代码,例如输入 `admin' OR '1'='1`,使得SQL查询返回更多数据,甚至可能泄露敏感信息。

\subsection{不安全的代码示例}
\begin{lstlisting}[language=Python, caption=存在SQL注入漏洞的代码]
import sqlite3

def get_user_data(username):
    conn = sqlite3.connect('users.db')
    cursor = conn.cursor()
    query = f"SELECT * FROM users WHERE username = '{username}'"
    cursor.execute(query)
    return cursor.fetchall()
\end{lstlisting}

\subsection{修复方案}
应使用参数化查询来避免直接将用户输入拼接到SQL查询中,从而防止SQL注入攻击。

\subsection{修正后的代码示例}
\begin{lstlisting}[language=Python, caption=使用参数化查询的安全代码]
def get_user_data(username):
    conn = sqlite3.connect('users.db')
    cursor = conn.cursor()
    query = "SELECT * FROM users WHERE username = ?"
    cursor.execute(query, (username,))
    return cursor.fetchall()
\end{lstlisting}

\section{跨站脚本攻击 (XSS)}

\subsection{漏洞描述}
跨站脚本攻击(XSS)是一种通过在网页中注入恶意脚本,从而在其他用户浏览网页时执行这些脚本的攻击方式。这可能导致用户的敏感信息泄露,甚至控制用户的会话。

\subsection{攻击方式}
攻击者可能在评论或输入字段中注入类似`<script>alert('Hacked!');</script>`的恶意JavaScript代码。

\subsection{不安全的代码示例}
\begin{lstlisting}[language=Python, caption=存在XSS漏洞的代码]
def display_comment(comment):
    return f"<p>{comment}</p>"
\end{lstlisting}

\subsection{修复方案}
应对用户输入进行HTML转义,以防止注入的脚本被浏览器执行。

\subsection{修正后的代码示例}
\begin{lstlisting}[language=Python, caption=经过HTML转义的安全代码]
import html

def display_comment(comment):
    safe_comment = html.escape(comment)
    return f"<p>{safe_comment}</p>"
\end{lstlisting}

\section{命令注入 (Command Injection)}

\subsection{漏洞描述}
命令注入是一种通过将恶意输入注入系统命令中,来执行未经授权的命令或操作的攻击方式。这种漏洞通常发生在应用程序直接调用系统命令时。

\subsection{攻击方式}
攻击者可能输入类似 `; rm -rf /` 的命令,导致删除服务器上的所有文件。

\subsection{不安全的代码示例}
\begin{lstlisting}[language=Python, caption=存在命令注入漏洞的代码]
import os

def delete_file(filename):
    os.system(f"rm {filename}")
\end{lstlisting}

\subsection{修复方案}
应使用安全的命令处理方法,例如通过`shlex.quote`对用户输入进行处理,以防止输入被解释为多个命令。

\subsection{修正后的代码示例}
\begin{lstlisting}[language=Python, caption=经过安全处理的命令执行代码]
import os
import shlex

def delete_file(filename):
    safe_filename = shlex.quote(filename)
    os.system(f"rm {safe_filename}")
\end{lstlisting}

\section{文件上传漏洞}

\subsection{漏洞描述}
文件上传漏洞是指攻击者通过上传恶意文件,并使其在服务器上执行的攻击方式。这可能导致服务器被攻击者控制或敏感数据泄露。

\subsection{攻击方式}
攻击者可能上传一个包含恶意代码的文件,例如PHP脚本,然后访问该文件以执行代码。

\subsection{不安全的代码示例}
\begin{lstlisting}[language=Python, caption=存在文件上传漏洞的代码]
def save_uploaded_file(file):
    with open(f"/uploads/{file.filename}", "wb") as f:
        f.write(file.read())
\end{lstlisting}

\subsection{修复方案}
应限制上传文件的类型,并确保文件名不包含路径信息,避免目录遍历攻击。

\subsection{修正后的代码示例}
\begin{lstlisting}[language=Python, caption=限制文件类型的安全代码]
import os

def save_uploaded_file(file):
    filename = os.path.basename(file.filename)
    if not filename.endswith(('.png', '.jpg', '.jpeg', '.gif', '.pdf')):
        raise ValueError("Unsupported file type")
    with open(f"/uploads/{filename}", "wb") as f:
        f.write(file.read())
\end{lstlisting}

\section{弱密码存储 (Insecure Password Storage)}

\subsection{漏洞描述}
如果密码以明文存储,一旦数据库被攻击者获取,所有用户的密码将立即被泄露。这是一个严重的安全隐患。

\subsection{攻击方式}
攻击者通过获取数据库中的明文密码,可以直接使用这些密码登录用户的其他账户(如果用户在多个地方使用相同的密码)。

\subsection{不安全的代码示例}
\begin{lstlisting}[language=Python, caption=明文存储密码的代码]
def store_password(username, password):
    with open("passwords.txt", "a") as f:
        f.write(f"{username}:{password}\n")
\end{lstlisting}

\subsection{修复方案}
应使用加密算法(如bcrypt)对密码进行哈希存储,而不是明文存储。

\subsection{修正后的代码示例}
\begin{lstlisting}[language=Python, caption=使用哈希函数存储密码的安全代码]
import bcrypt

def store_password(username, password):
    hashed = bcrypt.hashpw(password.encode(), bcrypt.gensalt())
    with open("passwords.txt", "a") as f:
        f.write(f"{username}:{hashed.decode()}\n")
\end{lstlisting}

\end{document}